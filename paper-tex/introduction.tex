\section{Pendahuluan}
Ketergantungan seseorang terhadap informasi tidak terlepas dari kebutuhan manusia akan informasi yang berada disekitarnya. Informasi yang diterima seseorang pada masa sekarang dapat melalui media fisik dan media digital. Media fisik seperti koran dan majalah, sedangkan media digital seperti facebook dan twitter. Media-media tersebut sanggup untuk menyebarkan informasi sangat cepat, sehingga orang-orang dengan cepat mengetahui informasi yang berada disekitarnya.
\\
\indent Informasi digital yang beredar di dunia maya pun tidak lepas dari penyalahgunaan informasi. Dibutuhkan suatu teknik penyandian terhadap data yang dimiliki agar data yang bersifat rahasia itu tidak diketahui dengan orang – orang yang tidak berkepentingan.Akan tetapi hal ini menarik perhatian dari pihak-pihak yang mengingkan informasi tersebut tetapi informasi yang diperoleh hanyalah terbatas dengan kepingan-kepingan saja. Seperti contohnya adalah studi kasus SPOJ \textit{The Bytelandian Cryptographer(Act IV)}. Pada studi kasus ini diketahui bahwa metode enskripsi yang digunakan adalah \textit{Vigenere Cipher}. Diberikan sejumlah kasus ujicoba, dimana pada setiap kasus ujicoba diberikan sejumlah potongan-potongan informasi dari \textit{plaintext} dan \textit{ciphertext}, batas atas dari panjang kunci yang digunakan(Panjang Kunci yang diberikan bukan panjang kunci yang sesungguhnya). Diharapkan dari studi kasus tersebut adalah merekonstruksi ulang \textit{plaintext} dari apa yang telah tersedia. Batasan masalahnya dimana jumlah inputan tidak akan melebihi dari 2 MB, batas atas panjang kuncinya bernilai $1<=M<=100,000$, dan jumlah kasus ujicoba $1<=T<=200$. Solusi dari studi kasus yang akan dibahas akan diimplementasikan dalam bentuk kode. 
\\
\indent Hasil dari Tugas Akhir ini diharapkan dapat merekonstruksi ulang pesan dari kepingan-kepingan informasi yang telah didapatakan sebelumnya. Sehingga diharapkan memberikan kontribusi pada pengembangan ilmu pengetahuan dan komunikasi. 
% \subsection{Subsection Heading Here}
% \blindtext
