
\section{Tinjuan Pustaka}

\subsection{\textit{Kasiski Examination}}
\textit{Kasiski Examination} adalah suatu teknik yang digunakan untuk mencari panjang kunci dari suatu \textit{ciphertext} dari suatu \textit{Polyalphabetic Cipher} dan turunannya. \textit{Kasiski Examination}  memanfaatkan kelemahan dari \textit{Polyalphebetical Cipher}, tanpa harus mengetahui \textit{plaintext} dan himpunan kunci yang digunakan. Kelemahannya yaitu apabila suatu \textit{substring} \textit{plaintext} yang sama dienskripsi dengan \textit{substring} dari kunci yang digunakan akan menghasilkan pola yang sama \cite{noauthor_kasiski_nodate}. Metode pencarian panjang kunci yang dilakukan adalaha sebagai berikut.
\begin{enumerate}
\item Mencari semua \textit{substring} yang berulang pada suatu kalimat.
\item Mencari panjang dimana \textit{subtring} tersebut berulang kembali.
\item Mencari semua faktor dari nilai yang diperoleh dari tahap dua.
\item Mencari faktor persekutuan terbesar dari hasil yang diperoleh pada tahap tiga.
\end{enumerate}


\subsection{\textit{Intersection}}
\textit{Intersection} adalah himpunan $A$ dan himpunan $B$, dimana ada bagian dari $A$ juga merupakan bagian dari $B$. \cite{devlin_joy_1993}.

