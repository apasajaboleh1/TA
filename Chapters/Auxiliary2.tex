\chapter{Hasil Percobaan dengan menggunakan Algoritma Naive dan Kasiski Examination}
\setcounter{table}{0}
  \renewcommand{\thetable}{B.\arabic{table}}
  \renewcommand{\thefigure}{B.\arabic{figure}}

\begin{table}[H]
\centering
\caption {Hasil Percobaan Penyelesaian Studi Kasus SPOJ The Bytelandian Cryptographer(Act IV) dengan menggunakan algoritma optimasi \textit{Kasiski Examination} dan \textit{Intersection} (1)}
\begin{tabular}{|c|c|c|}\hline
N&Waktu(Detik)&Waktu(log Detik)\\ \hline
10000&0.0156259&-1.80615\\ \hline
10100&0.0156259&-1.80615\\ \hline
10200&0.0156244&-1.8062\\ \hline
10300&0.0156244&-1.8062\\ \hline
10400&0.0156244&-1.8062\\ \hline
10500&0.0156255&-1.80617\\ \hline
10600&0.0156255&-1.80617\\ \hline
10700&0.0156255&-1.80617\\ \hline
10800&0.0156255&-1.80617\\ \hline
10900&0.0156255&-1.80617\\ \hline
11000&0.0156248&-1.80619\\ \hline
11100&0.0156351&-1.8059\\ \hline
11200&0.0156351&-1.8059\\ \hline
11300&0.0156351&-1.8059\\ \hline
11400&0.0156251&-1.80618\\ \hline
11500&0.0156251&-1.80618\\ \hline
11600&0.0156259&-1.80615\\ \hline
11700&0.0156259&-1.80615\\ \hline
11800&0.0156259&-1.80615\\ \hline
11900&0.0156259&-1.80615\\ \hline
\end{tabular}
\label{tab:res3}
\end{table}
\begin{table}[H]
\centering
\caption {Hasil Percobaan Penyelesaian Studi Kasus SPOJ The Bytelandian Cryptographer(Act IV) dengan menggunakan algoritma optimasi \textit{Kasiski Examination} dan \textit{Intersection} (2)}
\begin{tabular}{|c|c|c|}\hline
N&Waktu(Detik)&Waktu(log Detik)\\ \hline
12000&0.0156252&-1.80617\\ \hline
12100&0.0156252&-1.80617\\ \hline
12200&0.001048&-2.97964\\ \hline
12300&0.0156267&-1.80613\\ \hline
12400&0.0156267&-1.80613\\ \hline
12500&0.0156267&-1.80613\\ \hline
12600&0.0156237&-1.80622\\ \hline
12700&0.0156237&-1.80622\\ \hline
12800&0.0156244&-1.8062\\ \hline
12900&0.0156244&-1.8062\\ \hline
13000&0.0156244&-1.8062\\ \hline
13100&0.0156244&-1.8062\\ \hline
13200&0.0156244&-1.8062\\ \hline
13300&0.0156244&-1.8062\\ \hline
13400&0.0156229&-1.80624\\ \hline
13500&0.0156251&-1.80618\\ \hline
13600&0.0156441&-1.80565\\ \hline
13700&0.0156351&-1.8059\\ \hline
13800&0.0156351&-1.8059\\ \hline
13900&0.0156351&-1.8059\\ \hline
\end{tabular}
\label{tab:res5}
\end{table}
\begin{table}[H]
\centering
\caption {Hasil Percobaan Penyelesaian Studi Kasus SPOJ The Bytelandian Cryptographer(Act IV) dengan menggunakan algoritma optimasi \textit{Kasiski Examination} dan \textit{Intersection} (3)}
\begin{tabular}{|c|c|c|}\hline
N&Waktu(Detik)&Waktu(log Detik)\\ \hline
14000&0.0156149&-1.80646\\ \hline
14100&0.0156252&-1.80617\\ \hline
14200&0.0156252&-1.80617\\ \hline
14300&0.0156252&-1.80617\\ \hline
14400&0.0156343&-1.80592\\ \hline
14500&0.0156343&-1.80592\\ \hline
14600&0.0156343&-1.80592\\ \hline
14700&0.0156343&-1.80592\\ \hline
14800&0.0156229&-1.80624\\ \hline
14900&0.0156229&-1.80624\\ \hline
15000&0.0156229&-1.80624\\ \hline
15100&0.0156229&-1.80624\\ \hline
15200&0.0156229&-1.80624\\ \hline
15300&0.0156229&-1.80624\\ \hline
15400&0.0156354&-1.80589\\ \hline
15500&0.0156263&-1.80614\\ \hline
15600&0.0156263&-1.80614\\ \hline
15700&0.0156263&-1.80614\\ \hline
15800&0.0156263&-1.80614\\ \hline
15900&0.0156278&-1.8061\\ \hline
\end{tabular}
\label{tab:res7}
\end{table}
\begin{table}[H]
\centering
\caption {Hasil Percobaan Penyelesaian Studi Kasus SPOJ The Bytelandian Cryptographer(Act IV) dengan menggunakan algoritma optimasi \textit{Kasiski Examination} dan \textit{Intersection} (4)}
\begin{tabular}{|c|c|c|}\hline
N&Waktu(Detik)&Waktu(log Detik)\\ \hline
16000&0.0156256&-1.80616\\ \hline
16100&0.0156256&-1.80616\\ \hline
16200&0.0156256&-1.80616\\ \hline
16300&0.0156351&-1.8059\\ \hline
16400&0.0156244&-1.8062\\ \hline
16500&0.0156278&-1.8061\\ \hline
16600&0.0156278&-1.8061\\ \hline
16700&0.0156278&-1.8061\\ \hline
16800&0.0227098&-1.64379\\ \hline
16900&0.0156351&-1.8059\\ \hline
17000&0.0156153&-1.80645\\ \hline
17100&0.0156153&-1.80645\\ \hline
17200&0.0156153&-1.80645\\ \hline
17300&0.015635&-1.8059\\ \hline
17400&0.0156259&-1.80615\\ \hline
17500&0.0156347&-1.80591\\ \hline
17600&0.0156153&-1.80645\\ \hline
17700&0.0156153&-1.80645\\ \hline
17800&0.0156191&-1.80634\\ \hline
17900&0.0156247&-1.80619\\ \hline
\end{tabular}
\label{tab:res9}
\end{table}
\begin{table}[H]
\centering
\caption {Hasil Percobaan Penyelesaian Studi Kasus SPOJ The Bytelandian Cryptographer(Act IV) dengan menggunakan algoritma optimasi \textit{Kasiski Examination} dan \textit{Intersection} (5)}
\begin{tabular}{|c|c|c|}\hline
N&Waktu(Detik)&Waktu(log Detik)\\ \hline
18000&0.0156247&-1.80619\\ \hline
18100&0.0156247&-1.80619\\ \hline
18200&0.0156247&-1.80619\\ \hline
18300&0.0156229&-1.80624\\ \hline
18400&0.0156255&-1.80617\\ \hline
18500&0.0156247&-1.80619\\ \hline
18600&0.0156225&-1.80625\\ \hline
18700&0.0156225&-1.80625\\ \hline
18800&0.0156225&-1.80625\\ \hline
18900&0.0156225&-1.80625\\ \hline
19000&0.0156248&-1.80619\\ \hline
19100&0.0156275&-1.80611\\ \hline
19200&0.0156252&-1.80617\\ \hline
19300&0.0156206&-1.8063\\ \hline
19400&0.015626&-1.80615\\ \hline
19500&0.015626&-1.80615\\ \hline
19600&0.015626&-1.80615\\ \hline
19700&0.015626&-1.80615\\ \hline
19800&0.015626&-1.80615\\ \hline
19900&0.0156248&-1.80619\\ \hline
\end{tabular}
\label{tab:res11}
\end{table}


\begin{table}[H]
\centering
\caption {Hasil Percobaan Penyelesaian Studi Kasus SPOJ The Bytelandian Cryptographer(Act IV) dengan menggunakan algoritma \textit{Naive} (1)}
\begin{tabular}{|c|c|c|}\hline
N&Waktu(Detik)&Waktu(Log Detik)\\ \hline
10000&0.390619&-0.408247\\ \hline
10100&0.394834&-0.403585\\ \hline
10200&0.400778&-0.397096\\ \hline
10300&0.406246&-0.391211\\ \hline
10400&0.406737&-0.390686\\ \hline
10500&0.42187&-0.374821\\ \hline
10600&0.421361&-0.375346\\ \hline
10700&0.437496&-0.359026\\ \hline
10800&0.437661&-0.358862\\ \hline
10900&0.45313&-0.343777\\ \hline
11000&0.468254&-0.329519\\ \hline
11100&0.468746&-0.329062\\ \hline
11200&0.469241&-0.328604\\ \hline
11300&0.48438&-0.314814\\ \hline
11400&0.48388&-0.315262\\ \hline
11500&0.498918&-0.301971\\ \hline
11600&0.501698&-0.299558\\ \hline
11700&0.515621&-0.287669\\ \hline
11800&0.516122&-0.287248\\ \hline
11900&0.531256&-0.274696\\ \hline
\end{tabular}
\label{tab:res2}
\end{table}
\begin{table}[H]
\centering
\caption {Hasil Percobaan Penyelesaian Studi Kasus SPOJ The Bytelandian Cryptographer(Act IV) dengan menggunakan algoritma \textit{Naive} (2)}
\begin{tabular}{|c|c|c|}\hline
N&Waktu(Detik)&Waktu(Log Detik)\\ \hline
12000&0.530714&-0.275139\\ \hline
12100&0.546407&-0.262484\\ \hline
12200&0.54688&-0.262108\\ \hline
12300&0.578888&-0.237405\\ \hline
12400&0.562516&-0.249865\\ \hline
12500&0.593242&-0.226768\\ \hline
12600&0.594233&-0.226043\\ \hline
12700&0.593746&-0.226399\\ \hline
12800&0.608855&-0.215486\\ \hline
12900&0.621452&-0.206592\\ \hline
13000&0.640633&-0.193391\\ \hline
13100&0.640096&-0.193755\\ \hline
13200&0.641099&-0.193075\\ \hline
13300&0.656247&-0.182933\\ \hline
13400&0.687057&-0.163007\\ \hline
13500&0.687981&-0.162424\\ \hline
13600&0.687508&-0.162722\\ \hline
13700&0.702599&-0.153292\\ \hline
13800&0.703614&-0.152666\\ \hline
13900&0.718225&-0.143739\\ \hline
\end{tabular}
\label{tab:res4}
\end{table}
\begin{table}[H]
\centering
\caption {Hasil Percobaan Penyelesaian Studi Kasus SPOJ The Bytelandian Cryptographer(Act IV) dengan menggunakan algoritma \textit{Naive} (3)}
\begin{tabular}{|c|c|c|}\hline
N&Waktu(Detik)&Waktu(Log Detik)\\ \hline
14000&0.734383&-0.134077\\ \hline
14100&0.734832&-0.133812\\ \hline
14200&0.749211&-0.125396\\ \hline
14300&0.766334&-0.115582\\ \hline
14400&0.789337&-0.102738\\ \hline
14500&0.796879&-0.0986076\\ \hline
14600&0.81308&-0.0898667\\ \hline
14700&0.812038&-0.0904236\\ \hline
14800&0.813018&-0.0898998\\ \hline
14900&0.827625&-0.0821664\\ \hline
15000&0.844241&-0.0735336\\ \hline
15100&0.827581&-0.0821895\\ \hline
15200&0.843761&-0.0737806\\ \hline
15300&0.859857&-0.0655738\\ \hline
15400&0.874509&-0.0582357\\ \hline
15500&0.891106&-0.0500706\\ \hline
15600&0.890121&-0.050551\\ \hline
15700&0.906285&-0.0427352\\ \hline
15800&0.921589&-0.0354627\\ \hline
15900&0.922729&-0.0349258\\ \hline
\end{tabular}
\label{tab:res6}
\end{table}
\begin{table}[H]
\centering
\caption {Hasil Percobaan Penyelesaian Studi Kasus SPOJ The Bytelandian Cryptographer(Act IV) dengan menggunakan algoritma \textit{Naive} (4)}
\begin{tabular}{|c|c|c|}\hline
N&Waktu(Detik)&Waktu(Log Detik)\\ \hline
16000&0.937003&-0.028259\\ \hline
16100&0.952834&-0.0209828\\ \hline
16200&0.953624&-0.0206228\\ \hline
16300&1.04645&0.0197185\\ \hline
16400&1.10416&0.043032\\ \hline
16500&1.0375&0.0159881\\ \hline
16600&1.03005&0.0128583\\ \hline
16700&1.03179&0.0135913\\ \hline
16800&1.03075&0.0131533\\ \hline
16900&1.04765&0.0202162\\ \hline
17000&1.07823&0.0327114\\ \hline
17100&1.09324&0.0387155\\ \hline
17200&1.09423&0.0391086\\ \hline
17300&1.10882&0.0448611\\ \hline
17400&1.10902&0.0449394\\ \hline
17500&1.12558&0.0513764\\ \hline
17600&1.14016&0.0569658\\ \hline
17700&1.15626&0.0630555\\ \hline
17800&1.17237&0.0690647\\ \hline
17900&1.18701&0.0744544\\ \hline
\end{tabular}
\label{tab:res8}
\end{table}
\begin{table}[H]
\centering
\caption {Hasil Percobaan Penyelesaian Studi Kasus SPOJ The Bytelandian Cryptographer(Act IV) dengan menggunakan algoritma \textit{Naive} (5)}
\begin{tabular}{|c|c|c|}\hline
N&Waktu(Detik)&Waktu(Log Detik)\\ \hline
18000&1.18803&0.0748274\\ \hline
18100&1.20266&0.0801429\\ \hline
18200&1.21944&0.0861604\\ \hline
18300&1.23395&0.0912976\\ \hline
18400&1.24748&0.0960336\\ \hline
18500&1.25243&0.0977535\\ \hline
18600&1.26614&0.102482\\ \hline
18700&1.28063&0.107424\\ \hline
18800&1.2969&0.112906\\ \hline
18900&1.31404&0.118609\\ \hline
19000&1.32662&0.122747\\ \hline
19100&1.35917&0.133274\\ \hline
19200&1.34388&0.12836\\ \hline
19300&1.36163&0.134059\\ \hline
19400&1.37299&0.137667\\ \hline
19500&1.39379&0.144197\\ \hline
19600&1.40303&0.147067\\ \hline
19700&1.43746&0.157596\\ \hline
19800&1.43739&0.157575\\ \hline
19900&1.43803&0.157768\\ \hline
\end{tabular}
\label{tab:res10}
\end{table}