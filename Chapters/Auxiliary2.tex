\chapter{Hasil Percobaan dengan menggunakan Algoritma Naive dan Kasiski Examination}
\setcounter{table}{0}
  \renewcommand{\thetable}{B.\arabic{table}}
  \renewcommand{\thefigure}{B.\arabic{figure}}

\begin{table}[H]
\centering
\caption {Hasil Percobaan Penyelesaian Studi Kasus SPOJ The Bytelandian Cryptographer(Act IV) dengan menggunakan algoritma optimasi \textit{Kasiski Examination} dan \textit{Intersection} (1)}
\begin{tabular}{|c|c|c|}\hline
N&Waktu(Detik)&Waktu(log Detik)\\ \hline
10000&0.0156259&-1.80615\\ \hline
10100&0.0156259&-1.80615\\ \hline
10200&0.0156244&-1.8062\\ \hline
10300&0.0156244&-1.8062\\ \hline
10400&0.0156244&-1.8062\\ \hline
10500&0.0156255&-1.80617\\ \hline
10600&0.0156255&-1.80617\\ \hline
10700&0.0156255&-1.80617\\ \hline
10800&0.0156255&-1.80617\\ \hline
10900&0.0156255&-1.80617\\ \hline
11000&0.0156248&-1.80619\\ \hline
11100&0.0156351&-1.8059\\ \hline
11200&0.0156351&-1.8059\\ \hline
11300&0.0156351&-1.8059\\ \hline
11400&0.0156251&-1.80618\\ \hline
11500&0.0156251&-1.80618\\ \hline
11600&0.0156259&-1.80615\\ \hline
11700&0.0156259&-1.80615\\ \hline
11800&0.0156259&-1.80615\\ \hline
11900&0.0156259&-1.80615\\ \hline
\end{tabular}
\label{tab:res3}
\end{table}
\begin{table}[H]
\centering
\caption {Hasil Percobaan Penyelesaian Studi Kasus SPOJ The Bytelandian Cryptographer(Act IV) dengan menggunakan algoritma optimasi \textit{Kasiski Examination} dan \textit{Intersection} (2)}
\begin{tabular}{|c|c|c|}\hline
N&Waktu(Detik)&Waktu(log Detik)\\ \hline
12000&0.0156252&-1.80617\\ \hline
12100&0.0156252&-1.80617\\ \hline
12200&0.001048&-2.97964\\ \hline
12300&0.0156267&-1.80613\\ \hline
12400&0.0156267&-1.80613\\ \hline
12500&0.0156267&-1.80613\\ \hline
12600&0.0156237&-1.80622\\ \hline
12700&0.0156237&-1.80622\\ \hline
12800&0.0156244&-1.8062\\ \hline
12900&0.0156244&-1.8062\\ \hline
13000&0.0156244&-1.8062\\ \hline
13100&0.0156244&-1.8062\\ \hline
13200&0.0156244&-1.8062\\ \hline
13300&0.0156244&-1.8062\\ \hline
13400&0.0156229&-1.80624\\ \hline
13500&0.0156251&-1.80618\\ \hline
13600&0.0156441&-1.80565\\ \hline
13700&0.0156351&-1.8059\\ \hline
13800&0.0156351&-1.8059\\ \hline
13900&0.0156351&-1.8059\\ \hline
\end{tabular}
\label{tab:res5}
\end{table}
\begin{table}[H]
\centering
\caption {Hasil Percobaan Penyelesaian Studi Kasus SPOJ The Bytelandian Cryptographer(Act IV) dengan menggunakan algoritma optimasi \textit{Kasiski Examination} dan \textit{Intersection} (3)}
\begin{tabular}{|c|c|c|}\hline
N&Waktu(Detik)&Waktu(log Detik)\\ \hline
14000&0.0156149&-1.80646\\ \hline
14100&0.0156252&-1.80617\\ \hline
14200&0.0156252&-1.80617\\ \hline
14300&0.0156252&-1.80617\\ \hline
14400&0.0156343&-1.80592\\ \hline
14500&0.0156343&-1.80592\\ \hline
14600&0.0156343&-1.80592\\ \hline
14700&0.0156343&-1.80592\\ \hline
14800&0.0156229&-1.80624\\ \hline
14900&0.0156229&-1.80624\\ \hline
15000&0.0156229&-1.80624\\ \hline
15100&0.0156229&-1.80624\\ \hline
15200&0.0156229&-1.80624\\ \hline
15300&0.0156229&-1.80624\\ \hline
15400&0.0156354&-1.80589\\ \hline
15500&0.0156263&-1.80614\\ \hline
15600&0.0156263&-1.80614\\ \hline
15700&0.0156263&-1.80614\\ \hline
15800&0.0156263&-1.80614\\ \hline
15900&0.0156278&-1.8061\\ \hline
\end{tabular}
\label{tab:res7}
\end{table}
\begin{table}[H]
\centering
\caption {Hasil Percobaan Penyelesaian Studi Kasus SPOJ The Bytelandian Cryptographer(Act IV) dengan menggunakan algoritma optimasi \textit{Kasiski Examination} dan \textit{Intersection} (4)}
\begin{tabular}{|c|c|c|}\hline
N&Waktu(Detik)&Waktu(log Detik)\\ \hline
16000&0.0156256&-1.80616\\ \hline
16100&0.0156256&-1.80616\\ \hline
16200&0.0156256&-1.80616\\ \hline
16300&0.0156351&-1.8059\\ \hline
16400&0.0156244&-1.8062\\ \hline
16500&0.0156278&-1.8061\\ \hline
16600&0.0156278&-1.8061\\ \hline
16700&0.0156278&-1.8061\\ \hline
16800&0.0227098&-1.64379\\ \hline
16900&0.0156351&-1.8059\\ \hline
17000&0.0156153&-1.80645\\ \hline
17100&0.0156153&-1.80645\\ \hline
17200&0.0156153&-1.80645\\ \hline
17300&0.015635&-1.8059\\ \hline
17400&0.0156259&-1.80615\\ \hline
17500&0.0156347&-1.80591\\ \hline
17600&0.0156153&-1.80645\\ \hline
17700&0.0156153&-1.80645\\ \hline
17800&0.0156191&-1.80634\\ \hline
17900&0.0156247&-1.80619\\ \hline
\end{tabular}
\label{tab:res9}
\end{table}
\begin{table}[H]
\centering
\caption {Hasil Percobaan Penyelesaian Studi Kasus SPOJ The Bytelandian Cryptographer(Act IV) dengan menggunakan algoritma optimasi \textit{Kasiski Examination} dan \textit{Intersection} (5)}
\begin{tabular}{|c|c|c|}\hline
N&Waktu(Detik)&Waktu(log Detik)\\ \hline
18000&0.0156247&-1.80619\\ \hline
18100&0.0156247&-1.80619\\ \hline
18200&0.0156247&-1.80619\\ \hline
18300&0.0156229&-1.80624\\ \hline
18400&0.0156255&-1.80617\\ \hline
18500&0.0156247&-1.80619\\ \hline
18600&0.0156225&-1.80625\\ \hline
18700&0.0156225&-1.80625\\ \hline
18800&0.0156225&-1.80625\\ \hline
18900&0.0156225&-1.80625\\ \hline
19000&0.0156248&-1.80619\\ \hline
19100&0.0156275&-1.80611\\ \hline
19200&0.0156252&-1.80617\\ \hline
19300&0.0156206&-1.8063\\ \hline
19400&0.015626&-1.80615\\ \hline
19500&0.015626&-1.80615\\ \hline
19600&0.015626&-1.80615\\ \hline
19700&0.015626&-1.80615\\ \hline
19800&0.015626&-1.80615\\ \hline
19900&0.0156248&-1.80619\\ \hline
\end{tabular}
\label{tab:res11}
\end{table}


\begin{table}[H]
\centering
\caption {Hasil Percobaan Penyelesaian Studi Kasus SPOJ The Bytelandian Cryptographer(Act IV) dengan menggunakan algoritma \textit{Naive} (1)}
\begin{tabular}{|c|c|c|}\hline
N&Waktu(Detik)&Waktu(log Detik)\\ \hline
10000&0.390619\\ \hline
10100&0.394834\\ \hline
10200&0.400778\\ \hline
10300&0.406246\\ \hline
10400&0.406737\\ \hline
10500&0.42187\\ \hline
10600&0.421361\\ \hline
10700&0.437496\\ \hline
10800&0.437661\\ \hline
10900&0.45313\\ \hline
11000&0.468254\\ \hline
11100&0.468746\\ \hline
11200&0.469241\\ \hline
11300&0.48438\\ \hline
11400&0.48388\\ \hline
11500&0.498918\\ \hline
11600&0.501698\\ \hline
11700&0.515621\\ \hline
11800&0.516122\\ \hline
11900&0.531256\\ \hline
\end{tabular}
\label{tab:res2}
\end{table}
\begin{table}[H]
\centering
\caption {Hasil Percobaan Penyelesaian Studi Kasus SPOJ The Bytelandian Cryptographer(Act IV) dengan menggunakan algoritma \textit{Naive} (2)}
\begin{tabular}{|c|c|c|}\hline
N&Waktu(Detik)&Waktu(log Detik)\\ \hline
12000&0.530714\\ \hline
12100&0.546407\\ \hline
12200&0.54688\\ \hline
12300&0.578888\\ \hline
12400&0.562516\\ \hline
12500&0.593242\\ \hline
12600&0.594233\\ \hline
12700&0.593746\\ \hline
12800&0.608855\\ \hline
12900&0.621452\\ \hline
13000&0.640633\\ \hline
13100&0.640096\\ \hline
13200&0.641099\\ \hline
13300&0.656247\\ \hline
13400&0.687057\\ \hline
13500&0.687981\\ \hline
13600&0.687508\\ \hline
13700&0.702599\\ \hline
13800&0.703614\\ \hline
13900&0.718225\\ \hline
\end{tabular}
\label{tab:res4}
\end{table}
\begin{table}[H]
\centering
\caption {Hasil Percobaan Penyelesaian Studi Kasus SPOJ The Bytelandian Cryptographer(Act IV) dengan menggunakan algoritma \textit{Naive} (3)}
\begin{tabular}{|c|c|c|}\hline
N&Waktu(Detik)&Waktu(log Detik)\\ \hline
14000&0.734383\\ \hline
14100&0.734832\\ \hline
14200&0.749211\\ \hline
14300&0.766334\\ \hline
14400&0.789337\\ \hline
14500&0.796879\\ \hline
14600&0.81308\\ \hline
14700&0.812038\\ \hline
14800&0.813018\\ \hline
14900&0.827625\\ \hline
15000&0.844241\\ \hline
15100&0.827581\\ \hline
15200&0.843761\\ \hline
15300&0.859857\\ \hline
15400&0.874509\\ \hline
15500&0.891106\\ \hline
15600&0.890121\\ \hline
15700&0.906285\\ \hline
15800&0.921589\\ \hline
15900&0.922729\\ \hline
\end{tabular}
\label{tab:res6}
\end{table}
\begin{table}[H]
\centering
\caption {Hasil Percobaan Penyelesaian Studi Kasus SPOJ The Bytelandian Cryptographer(Act IV) dengan menggunakan algoritma \textit{Naive} (4)}
\begin{tabular}{|c|c|c|}\hline
N&Waktu(Detik)&Waktu(log Detik)\\ \hline
16000&0.937003\\ \hline
16100&0.952834\\ \hline
16200&0.953624\\ \hline
16300&1.04645\\ \hline
16400&1.10416\\ \hline
16500&1.0375\\ \hline
16600&1.03005\\ \hline
16700&1.03179\\ \hline
16800&1.03075\\ \hline
16900&1.04765\\ \hline
17000&1.07823\\ \hline
17100&1.09324\\ \hline
17200&1.09423\\ \hline
17300&1.10882\\ \hline
17400&1.10902\\ \hline
17500&1.12558\\ \hline
17600&1.14016\\ \hline
17700&1.15626\\ \hline
17800&1.17237\\ \hline
17900&1.18701\\ \hline
\end{tabular}
\label{tab:res8}
\end{table}
\begin{table}[H]
\centering
\caption {Hasil Percobaan Penyelesaian Studi Kasus SPOJ The Bytelandian Cryptographer(Act IV) dengan menggunakan algoritma \textit{Naive} (5)}
\begin{tabular}{|c|c|c|}\hline
N&Waktu(Detik)&Waktu(log Detik)\\ \hline
18000&1.18803\\ \hline
18100&1.20266\\ \hline
18200&1.21944\\ \hline
18300&1.23395\\ \hline
18400&1.24748\\ \hline
18500&1.25243\\ \hline
18600&1.26614\\ \hline
18700&1.28063\\ \hline
18800&1.2969\\ \hline
18900&1.31404\\ \hline
19000&1.32662\\ \hline
19100&1.35917\\ \hline
19200&1.34388\\ \hline
19300&1.36163\\ \hline
19400&1.37299\\ \hline
19500&1.39379\\ \hline
19600&1.40303\\ \hline
19700&1.43746\\ \hline
19800&1.43739\\ \hline
19900&1.43803\\ \hline
\end{tabular}
\label{tab:res10}
\end{table}