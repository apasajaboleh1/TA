\chapter{PENDAHULUAN}
  Pada bab ini akan dipaparkan mengenai garis besar Tugas Akhir yang meliputi latar belakang, tujuan, rumusan dan batasan permasalahan, metodologi pembuatan Tugas Akhir, dan sistematika penulisan.
  
  \section{Latar Belakang}	
	\indent Ketergantungan seseorang terhadap informasi tidak terlepas dari kebutuhan manusia akan informasi yang berada di sekitarnya. Informasi yang diterima seseorang pada masa sekarang dapat melalui media fisik dan media digital. Media fisik seperti koran dan majalah, sedangkan media digital seperti Facebook dan Twitter. Media-media tersebut sanggup untuk menyebarkan informasi dengan sangat cepat, sehingga orang-orang dengan cepat mengetahui informasi yang berada di sekitarnya.
    \\
    \indent Pada zaman modern ini suatu informasi, terutama yang bersifat rahasia menjadi semakin rentan akan penyalahgunaan informasi tersebut. Oleh karena itu, informasi ini akan disimpan pada tempat-tempat yang aman dan penulisan dari informasi ini pada umumnya menggunakan sandi yang hanya dimengerti oleh orang-orang yang berkepentingan terhadap informasi tersebut.
    \\
	\indent Informasi digital yang beredar di dunia maya pun tidak lepas dari penyalahgunaan informasi. Dibutuhkan suatu teknik penyandian terhadap data yang dimiliki agar data yang bersifat rahasia itu tidak diketahui dengan orang–orang yang tidak berkepentingan. Teknik penyandian terhadap data digital dapat dibagi menjadi dua jika melihat dari teknik penyandiannya yaitu \textit{symmetric cipher} dan \textit{asymmetric cipher}. Teknik \textit{symmetric cipher} dapat dibagi menjadi menjadi empat bagian jika dilihat dari penyubtitusiannya yaitu \textit{Caesar cipher},\textit{ monoalphabetic cipher}, \textit{polyalphabetic cipher}, dan \textit{one time pad}. Pada dasarnya pendeskripsian dari data yang terenkripsi dengan penyandian \textit{symmetric cipher} dengan cara mengetahui kuncinya dan tipe dari penyubtitusiannya.
	\\
	\indent Dalam Tugas Akhir ini penulis akan mencoba mendeskripsikan informasi terbut dengan menggunakan metode \textit{symmetric cipher} dan teknik subtitusinya menggunakan \textit{polyalphabetic cipher}. Salah satunya dengan menggunakan modifikasi \textit{Kasiski Examination}, akan tetapi dalam permasalahan ini apabila hanya menggunakan \textit{Kasiski Examination} waktu yang dibutuhkan sangatlah besar, oleh karena itu penulis mengoptimasi metode yang telah ada.
    
  \section{Rumusan Masalah}
    Rumusan masalah yang diangkat dalam tugas akhir ini adalah sebagai berikut: 
    \begin{enumerate}
      \item Bagaimana penerapan Optimasi \textit{Kasiski Examination} untuk menyelesaikan studi kasus SPOJ \textit{The Bytelandian Cryptographer(Act IV)}?
      \item Bagaimana hasil dari kinerja Optimasi \textit{Kasiski Examination} yang digunakan untuk menyelesaikan studi kasus SPOJ \textit{The Bytelandian Cryptographer(Act IV)}?
    \end{enumerate}

  \section{Batasan Masalah}
  	\label{batasan-masalah}
    Dari permasalahan yang telah diuraikan di atas, terdapat beberapa batasan masalah pada tugas akhir ini, yaitu:
    \begin{enumerate}
      \item Bahasa pemrograman yang akan digunakan adalah bahasa pemrograman C/C++.
      \item Batasan maksimum panjang dari \textit{input file} sebesar $2$ MB.	
      \item Batasan maksimum panjang dari batas atas \textit{key} sebesar $100,000$ karakter.
      \item \textit{Dataset} yang digunakan adalah \textit{dataset} pada problem SPOJ \textit{The Bytelandian Cryptographer (Act IV)}.
    \end{enumerate}

  \section{Tujuan}
  \label{tujuan}
    Tujuan dari pengerjaan Tugas Akhir ini adalah: 
    \begin{enumerate}
      \item Menerapkan Optimasi \textit{Kasiski Examination} untuk menyelesaikan studi kasus SPOJ \textit{The Bytelandian Cryptographer (Act IV)}.
      \item Mengevalusi hasil dan kinerja Optimasi \textit{Kasiski Examination} dalam menyelesaikan studi kasus SPOJ \textit{The Bytelandian Cryptographer(Act IV)}
    \end{enumerate}
    
    
    \section{Metodologi}
    \label{metodologi}
	Langkah-langkah yang ditempuh dalam pengerjaan Tugas Akhir ini yaitu:
    \begin{enumerate}
    	\item \textbf{Penyusunan proposal Tugas Akhir} \\
		       Pada tahap ini dilakukan penyusunan proposal Tugas Akhir yang berisi permasalahan dan gagasan solusi yang akan diteliti pada SPOJ \textit{The Bytelandian Cryptographer (Act IV)}.
    	\item \textbf{Studi literatur}\\
		    	Pada tahap ini dilakukan pencarian informasi dan studi literatur mengenai pengetahuan atau metode yang dapat digunakan dalam penyelesaian masalah. Informasi didapatkan dari materi-materi yang berhubungan dengan algoritma yang digunakan untuk penyelesaian permasalahan ini, materi-materi tersebut didapatkan dari buku, jurnal, maupun internet.
    	\item \textbf{Desain}\\
		    	Pada tahap ini dilakukan desain rancangan algoritma yang digunakan dalam solusi untuk pemecahan SPOJ \textit{The Bytelandian Cryptographer (Act IV)} 
    	\item \textbf{Implementasi perangkat lunak}\\
		    	Pada tahap ini dilakukan implementasi atau realiasi dari rancangan desain algoritma yang telah dibangun pada tahap desain ke dalam bentuk program.
		 \item \textbf{Uji coba dan evaluasi}\\
		 Pada tahap ini dilakukan uji coba kebenaran implementasi. Pengujian kebenaran dilakukan pada sistem penilaian daring SPOJ sesuai dengan masalah yang dikerjakan untuk diuji apakah luaran dari program telah sesuai.
		 \item \textbf{Penyusunan buku Tugas Akhir}
		  Pada tahap ini dilakukan penyusunan buku Tugas Akhir yang berisi dokumentasi hasil pengerjaan Tugas Akhir.
	   	\end{enumerate}
	   	
    \section{Sistematika Penulisan}	
	    Buku Tugas Akhir ini bertujuan untuk mendapatkan gambaran dari pengerjaan Tugas Akhir ini. Secara garis besar, buku Tugas Akhir terdiri atas beberapa bagian seperti berikut ini:
	    \begin{labeling}{\textbf{Bab III}}
	    	\item[\textbf{Bab I}] 
		    	\textbf{Pendahuluan}\\
				Bab ini berisi latar belakang masalah, tujuan dan manfaat pembuatan Tugas Akhir, permasalahan, batasan masalah, metodologi yang digunakan, dan sistematika penyusunan Tugas Akhir. 
	    	\item[\textbf{Bab II}] \textbf{Dasar Teori} \\
		    	Bab ini berisi dasar teori mengenai permasalahan dan garis besar penyelesaian yang digunakan dalam Tugas Akhir dan deskripsi permasalahan yang digunakan dalam Tugas Akhir.
	    	\item[\textbf{Bab III}] \textbf{Desain} \\ 
		    	Bab ini berisi desain algoritma yang digunakan dalam penyelesaian permasalahan.
	    	\item[\textbf{Bab IV}] \textbf{Implementasi} \\
		    	Bab ini berisi implementasi berdasarkan desain algortima yang telah dilakukan pada tahap desain.	
	    	\item[\textbf{Bab V}] \textbf{Pengujian dan Evaluasi} \\
		    	Bab ini berisi uji coba dan evaluasi dari hasil implementasi yang telah dilakukan pada tahap implementasi.
	    	\item[\textbf{Bab VI}] 	\textbf{Kesimpulan dan Saran}\\
    	Bab ini berisi kesimpulan dari hasil pengujian yang dilakukan, dan membahas saran untuk pengembangan algoritma lebih lanjut. \\
\textbf{Daftar Pustaka}\\
Merupakan daftar referensi yang digunakan untuk mengembangkan Tugas Akhir.\\
		\textbf{Lampiran}\\
		Merupakan bab tambahan yang berisi hal-hal terkait yang penting dalam aplikasi ini.
		\end{labeling}