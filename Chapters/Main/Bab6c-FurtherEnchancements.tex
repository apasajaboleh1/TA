	\section{\textit{Further Enchancements}}
	Karena topik lelang \textit{online} adalah topik yang sangat luas dan penulis melihat masih sangat banyak sekali hal yang dapat dieksplor didalamnya, antara lain sebagai berikut:
	\begin{enumerate}
		\item \textbf{\textit{Advanced Data Management \& Searching Techniques}} \\
			Tokopedia - dalam blog \textit{tech}nya (tech.tokopedia.com) - pernah mempublikasi proses, kesulitan dan dampak perubahan migrasi penggunaan Apache Solr menjadi menggunakan ElasticSearch, salah satu \textit{tools powerful} yang juga belakangan ini sangat sering dibicarakan di forum developer online. Tokopedia yang menyimpan sangat banyak data barang, pasti memiliki metode tertentu untuk \textit{data searching} dan \textit{preprocessing}/\textit{data management} sedemikian rupa sehingga Tokopedia dapat tetap mempertahankan reputasinya sebagai \textit{e-commerce} terbaik dan bahkan reputasinya mengalahkan Wikipedia dan Twitter. Tokopedia sadar betul bahwa performa kecepatan website sangat penting dalam bisnis, namun mereka dapat mengatasi hal tersebut sekalipun data center mereka yang sudah sangat besar (mencapai 16,5 juta transaksi setiap bulannya pada tahun ketujuh). Tentu sangat menarik untuk menganalisa bagaimana Tokopedia berhasil menyelesaikan masalah tersebut.
		\item \textbf{\textit{User Experience's Impact \& Analysis to Bussiness Things}} \\
			Setelah penulis menganalisa \textit{user experience} lebih lanjut, ternyata ada sebuah fakta yang lebih menarik: \textit{search engine rankings} juga didasarkan pada \textit{usability, user experience dan content affect} dalam sebuah website. Hal ini tentu menjadi sebuah masalah, namun juga sebuah tantangan karena ini memaksa \textit{developer} untuk tidak hanya sekedar membuat \textit{website}, tapi juga membuat \textit{website} yang khusus didesain, \textit{tailored} dan \textit{crafted} kepada penggunanya. Tentu saja, topik ini sangat menarik untuk dianalisa dan dicari \textit{key-key point} yang berpengaruh besar terhadap \textit{search engine rankings}, yang pastinya sangat dicari oleh perusahaan-perusahaan bisnis.
			
		\item \textbf{\textit{Early Fraud Detection \& Credit Scoring for Item Suggestion using Machine Learning Techniques}}\\
			Setiap sistem yang dibangun, pasti mencatat \textit{log} dan menyimpannya untuk keperluan \textit{monitoring}, \textit{traceback} jika masalah terjadi, dan \textit{information gathering}. Hal ini menarik karena jika dianalisa secara pintar, kita bisa mendapatkan informasi menarik seperti \textit{early fraud detection} dengan cara menganalisa \textit{pattern} pengguna-pengguna yang terbukti melakukan \textit{fraud}, dan menggunakan \textit{pattern} tersebut untuk mengenali apakah perilaku pengguna tertentu sangat dekat dengan \textit{pattern fraud} yang sudah ditemukan. \\Selain itu, hal lain adalah penggunaan \textit{machine learning} untuk \textit{credit scoring} pengguna yang akan berpengaruh terhadap \textit{item suggestion}. Jika seorang pengguna dikenal baik (tidak terdeteksi \textit{pattern}/ tidak melakukan \textit{fraud}), dan sering \textit{purchase} barang-barang mahal ataupun berharga seperti surat tanah, ijin usaha dan lain-lain, tentu saja ini menandakan bahwa pengguna/\textit{customer} ini sifatnya loyal (\textit{scoringnya} baik/tinggi), tentu saja lebih baik jika kita rekomendasikan juga barang-barang yang dari pengguna yang \textit{scoring} dari yang tinggi pula, sehingga \textit{market}/pasar yang sudah dibangun di awal tetap terjaga seiring dengan perkembangan aplikasi.
	\end{enumerate}
