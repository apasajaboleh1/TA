\mychapter{0}{KATA PENGANTAR}
    
	  Puji Syukur kepada Tuhan yang Maha Esa, atas berkatNya penulis dapat menyelesaikan buku berjudul \textbf{\judul}. 
	  \newline
	  \indent Selain itu, pada kesempatan ini penulis menghaturkan terima kasih sebesar-besarnya kepada pihak-pihak yang tanpa mereka, penulis tidak akan dapat menyelesaikan buku ini:
  \begin{enumerate}
  	\item \textbf{\textit{Tuhan Yesus Kristus}}- atas segala berkat, limpahan karunia, kesempatan dan rancangan jalanNya-lah penulis masih diberi nafas kehidupan, waktu, tenaga dan pikiran untuk menyelesaikan buku ini.
    \item \textbf{Alm. Papa} yang selalu menguatkan, menasehati, dan luar biasa sabar dalam mengingatkan penulis agar tidak lupa menjaga kesehatan dan selalu bersyukur selama masa studi.
    \item \textbf{Mama dan saudara} yang selalu memberikan saran, dukungan, doa dan tidak lupa untuk selalu bersyukur selama masa studi.
    \item \textbf{Yth Bapak Rully Soelaiman} sebagai dosen pembimbing I yang telah banyak memberikan ilmu, bimbingan, nasihat, motivasi, serta waktu diskusi sehingga penulis dapat menyelesaikan tugas akhir ini; dan \\
	    \textbf{Yth Ibu Wijayanti Nurul Khotimah} sebagai dosen pembimbing II yang memberi bimbingan, saran teknis dan administratif, diskusi dan pemecahan masalah dalam pembuatan dan penulisan buku tugas akhir.
    \item \textbf{Teman-teman Sarjana Komedi} yang telah mengingatkan, memberikan semangat dan inspirasi untuk terus melanjutkan tugas akhir di saat penulis kehilangan semangat.
    \item \textbf{Teman-teman S1 Teknik Informatika 2013} yang membantu, menyemangati dan bertukar pikiran dengan  penulis selama pengerjaan tugas akhir ini.
    \item \textbf{Teman-teman S1 Teknik Informatika bukan 2013}, yang telah banyak membantu, menyemangati dan bertukar pikiran dengan penulis selama pengerjaan tugas akhir ini, terutama pada Steven, Theo, Daniel, dan Glenn.
    \item Serta semua pihak yang tidak tertulis, baik yang membantu dalam proses pengujian, membantu memikir saat ada masalah, dan lainnya yang telah turut membantu penulis dalam menyelesaikan Tugas Akhir ini.
  \end{enumerate}
  
  \indent Penulis menyadari bahwa Tugas Akhir ini masih memiliki banyak kekurangan. Oleh karena itu, penulis berharap kritik dan saran dari pembaca sekalian untuk memperbaiki buku ini ke depannya. Semoga tugas akhir ini dapat memberikan manfaat yang sebaik-baiknya.

  \hfill Surabaya, Nopember 2017 \\ \\ 


  \hfill Freddy Hermawan Yuwono

\cleardoublepage % Mengisi penanda halaman genap yang kosong

