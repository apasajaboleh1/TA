\chapter{PENUTUP}
  Bab ini membahas kesimpulan yang dapat diambil dari tujuan pembuatan sistem dan hubungannya dengan hasil uji coba yang telah dilakukan. Selain itu, terdapat beberapa saran yang bisa dijadikan acuan untuk melakukan pengembangan dan penelitian lebih lanjut.
  \section{Kesimpulan}
 Dari hasil uji coba yang telah dilakukan terhadap perancangan dan implementasi algoritma untuk menyelesaikan SPOJ \textit{The Bytelandian Cryptographer (Act IV)} dapat diambil kesimpulan sebagai berikut:
 
 \begin{enumerate}
 \item Implementasi algoritma dengan menggunakan teknik \textit{Kasiski Examnination} dengan adanya optimasi saja tidak dapat menyelesaikan permasalahan SPOJ \textit{The Bytelandian Cryptographer (Act IV)} dengan benar. Akan tetapi apabila ditambahkan metode \textit{Intersection} didalam teknik \textit{Kasiski Examnination} ditambah dengan optimasi dapat menyelesaikan permasalahan SPOJ \textit{The Bytelandian Cryptographer (Act IV)} dengan benar.
 \item Kompleksitas waktu $\mathcal{O}(T*\frac{M}{2}*(N+S))$ masih dapat menyelesaikan permasalahan SPOJ \textit{The Bytelandian Cryptographer (Act IV)} dalam rentang waktu yang telah ditetapkan.
 \item Waktu yang dibutuhkan oleh program untuk menyelesaikan SPOJ \textit{The Bytelandian Cryptographer (Act IV)} minimum $4,38$ detik, maksimum $4,49$ detik dan rata-rata $4.418$ detik. Memori yang dibutuhkan berkisar antara 26-27 MB. %Dibandingkan dengan menggunakan algoritma \textit{Naive} yang memakan waktu jauh lama. Perbandingannya dapat dilihat pada gambar \ref{fig:banding}.
 \end{enumerate}
  
  \section{Saran}
  Pada Tugas Akhir kali ini tentunya terdapat kekurangan serta nilai-nilai yang dapat penulis ambil. Berikut adalah saran-saran yang dapat diambil melalui Tugas Akhir ini:
  \begin{enumerate}
    \item Teknik \textit{Kasiski Examination} dalam pencarian panjang kunci masih cenderung lambat. Hal ini terjadi karena masih menggunakan teknik \textit{brute force}. Sehingga \textit{running time} yang diperoleh kurang optimal. Perlu adanya optimisasi lanjutan atau penggantian metode yang dapat mencari suatu panjang kunci lebih cepat. Melihat pada Gambar \ref{fig:per1} dan \ref{fig:per2} terdapat yang program yang memiliki \textit{running time} yang lebih cepat.%sebenarnya telah dilakukan ujicoba dengan menggunakan bilangan komposit dan bilangan prima yang telah terbentuk dalam persamaan ini, akan tetapi masih menemukan jalan buntu.
    %\item Perlu adanya Optimisasi dalam hal pencarian suatu indeks yang perlu dirubah atau tidak. Dengan teknik yang dipakai oleh penulis tidak dapat memenuhi ekspetasi jika berharap dengan hasil yang sangat cepat.
    
  \end{enumerate}
  
