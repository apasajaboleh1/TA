\begin{abstrak}
		\indent Pada era digitalisasi ini, tingkat kebutuhan masyarakat akan informasi semakin meningkat. Hal ini menyebabkan pertukaran informasi menjadi sangat mudah. Hal ini membuat informasi yang bersifat sensitif dapat terjadi kebocoran informasi kepada pihak-pihak yang tidak berkepentingan. Kebocoran informasi terbagi menjadi dua apabila dilihat dari keutuhan informasi yang didapat, yaitu sebagian dan seutuhnya. Kebocoran informasi yang bersifat sebagian, membuat pihak-pihak yang tidak berkepentingan tetapi yang menginginkan informasi tersebut, berusaha untuk mendapatkan informasi yang utuh dari potongan-potongan informasi yang telah didapatkan. 
\\
\indent Permasalahan dalam buku tugas akhir ini adalah permasalahan untuk mendapatkan \textit{plaintext} sebanyak-banyaknya dari \textit{ciphertext} dengan diberikan batas atas panjang kunci. Metode enkripsi yang digunakan adalah  teknik \textit{Vigenere Cipher}. Dalam permasalahan ini diberikan \textit{plaintext} dan \textit{ciphertext}, akan tetapi terdapat informasi yang hilang pada keduanya. Batas atas panjang kunci tersebut belum tentu panjang kunci yang sesungguhnya. Untuk dapat merekonstruksi \plaintext dari kepingan informasi yang didapatkan. Diperlukan untuk mencari panjang kunci yang didapatkan dengan cara memodifikasi \textit{Kasiski Examination} dan \textit{Intersection}. Beberapa hal yang perlu diperhatikan seperti mempercepat \textit{Kasiski Examination} dan juga \textit{Intersection} terhadap hasil yang diperoleh dari pencarian panjang kunci.
\\  
\indent Hasil dari tugas akhir ini telah berhasil untuk menyelesaikan permasalahan yang telah diangkat dengan benar. Waktu yang diperlukan untuk dapat menyelesaikan masukan sebesar 2MB adalah 4,42 detik dengan alokasi memori sebesar 26,5MB.
\\
\noindent \textbf{Kata-Kunci}: \textbf{ciphertext}, \textbf{Intersection}, \textbf{Kasiski Examination}, \textbf{optimisasi}, \textbf{plaintext}.
\end{abstrak}


