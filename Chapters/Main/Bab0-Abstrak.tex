\begin{abstrak}
		\indent Diberikan batas atas panjang kunci untuk mendeskripsikan \ciphertext. Diberikan \plaintext dan \ciphertext dengan ketentuan \ciphertext dan \plaintext yang didapatkan tidak memiliki bagian yang utuh. Tentukan \plaintext yang dapat direkonstruksi ulang sebanyak-banyaknya.
\\
\indent \textit{Kasisiki Examination} merupakan suatu teknik untuk mencari panjang kunci yang sesungguhnya yang bisa didapatkan. Diperlukan optimisasi pada \textit{Kasiski Examination} agar pencarian yang diperoleh bisa lebih cepat.
\\
\indent Pada tugas akhir ini akan merancang penyelesaian masalah yang disampaikan pada paragraf pertama dengan menggunakan \textit{Kasisiki Examination} yang dioptimasi.
\\
\noindent \textbf{Kata-Kunci}: \textit{plaintext}, \textit{ciphertext}, \textit{Kasiski Examination}, \textit{optimasi}.
\end{abstrak}


