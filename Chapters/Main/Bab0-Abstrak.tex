\begin{abstrak}
		\indent Pada Era Digitalisasi ini, tingkat kebutuhan masyarakat akan informasi semakin meningkat. Hal ini menyebabkan pertukaran informasi menjadi sangat mudah. Hal ini membuat informasi yang bersifat sensitif dapat terjadi kebocoran informasi kepada pihak - pihak yang tidak berkepentingan. Kebocoran informasi terbagi menjadi 2 apabila dilihat dari keutuhan informasi yang didapat, yaitu sebagian dan seutuhnya. Kebocoran informasi yang bersifat sebagian, membuat pihak-pihak yang tidak berkepentingan tetapi yang meminginkan informasi tersebut, berusaha untuk mendapatkan informasi yang utuh dari potongan-potongan informasi yang telah didapatkan. 
\\
\indent Permasalahan dalam buku tugas akhir ini adalah permasalahan untuk mendapatkan \plaintext sebanyak-banyaknya dari \ciphertext dan batas atas panjang kunci pada metode enskripsi yang menggunakan teknik \textit{Vigenere Cipher}. Dalam permasalahan ini diberikan \plaintext dan \ciphertext, akan tetapi terdapat informasi yang hilang pada keduanya. Diberikan batas atas panjang kunci, dimana batas atas ini belum tentu panjang kunci yang sesungguhnya.
\\  
\indent Tugas akhir ini mengimplemntasikan \textit{Kasiski Examination} yang dimodifikasi untuk dapat menyelesaikan permasalahan yang ada, tetapi membutuhkan waktu yang sangat lama. Perlu adanya optimasi yang digunakan untuk mempercepat solusi yang ada dengan menggunakan \textit{prunning} pada proses \textit{Kasiski Examination}.
\\
\noindent \textbf{Kata-Kunci}:  \textit{Plaintext}, \textit{Ciphertext}, \textit{Kasiski Examination},  \textit{Optimisasi}.
\end{abstrak}


